\documentclass{article}
\begin{document}
\begin{center} {\scshape\LARGE Detección de melanoma mediante segmentación semántica \par} \end{center}


\section*{Justificación}
El melanoma es un padecimiento que se puede tratar fácilmente cuando es detectado en una fase temprana, de no ser así el riesgo de complicación aumenta de forma exponencial al comenzar a expandirse rápidamente a otros tejidos. Por lo tanto, es indispensable desarrollar métodos de detección del padecimiento al alcance de las personas, uno de estos métodos sería la inteligencia artificial la cual nos permite predecir o reconstruir información a partir de una base de datos históricos que funcionan a modo de entrenamiento.

\section*{Hipótesis}
Las técnicas aplicadas para las tareas de segmentación semántica actuales son desarrolladas mediante un proceso lineal de transformaciones que requiere de ajustes manuales para dar aproximarse al resultado deseado. Mediante las redes neuronales convolucionales es posible crear una secuencia alternativa de transformaciones de forma automática utilizando un conjunto de datos correspondiente a la imagen entrante y la salida esperada, tras el entrenamiento se obtiene un modelo que permita generar las máscaras de segmentación semántica correspondientes al melanoma.

\section*{Objetivo}
Implementar una arquitectura de red neuronal convolucional para realizar el entrenamiento de modelos y de esta forma realizar las tareas de segmentación semántica en imágenes dermatológicas relacionadas al melanoma de piel.

\section*{Metodología}
El desarrollo de la tesis está seccionada en:

\begin{description}
\item[Investigación]{Hace un análisis del problema que se pretende resolver, define el objetivo general y específicos que deben cumplir la tesis.}

\item[Implementación]{Desarrolla e implementa la solución, utilizándola sobre contextos en los que funcionaría la solución desarrollada. }

\item[Evaluación]{Diseña y prueba experimentos para analizar el comportamiento de la solución propuesta en distintos contextos.}

\end{description}
\pagenumbering{gobble}
\end{document}
