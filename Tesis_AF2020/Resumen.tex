%Resumen

\chapter{Resumen}
\markboth{Resumen}{}

{\renewcommand{\baselinestretch}{1.1}\selectfont
{\setlength{\leftskip}{10mm}
\setlength{\parindent}{-10mm}

\autor.

Candidato para obtener el grado de \grado\orientacion.

\uanl.

\fime.

Título del estudio: \textsc{\titulo}.

\noindent Número de páginas: \pageref*{lastpage}.}

%%% Comienza a llenar aquí
\paragraph{Objetivos y método de estudio:}
El objetivo de el presente trabajo de tesis el uso de la tecnología de redes neuronales profundas (\emph{deep learning}) para la detección de melanoma de piel, mediante el uso de las redes neuronales profundas es posible desarrollar modelos cuyo entrenamiento dá como resultado un modelo que realiza una secuencia específica de transformaciones para convertir el dato de entrada en la salida especificada. El dato de entrada en este caso consta de tres matríces de intensidad de pixeles que corresponden a los tres canales de la imágen a color, y el dato de salida es un mapa binario correspondiente al mapa de segmentación en el cuál se remarcan las regiones de la imagen que corresponden a la categoría clasificada (tejido sano, melanoma). Para obtener el modelo que realiza la secuencia de transformaciones en dos fases: \emph{reducción}, \emph{reconstrucción}. La fase de \emph{reducción} trata de reducir la dimension de la imagen de entrada, conservando su información característica y posteriormente codificar dicha información en un dato multi-dimensional. La fase de \emph{reconstrucción}, corresponde a la reconstrucción de la imagen basándose en una predicción obtenida mediante los datos codificados en la \emph{reducción} y el escalado del mapa binario obtenido para crear la máscara de segmentación equivalente a las regiones clasificadas de la imagen.

\paragraph{Contribuciones y conclusiones:}
La principal contribución del presente trabajo de tesis es la de establecer los pasos requeridos para llevar a cabo el proceso de implementación de redes neuronales profundas para realizar tareas de segmentación semántica. El lenguaje utilizado en esta implementación fue \emph{Python} debido a que sus librerías permiten desplegar ágilmente modelos neuronales y utilizar métricas que permitan evaluar y experimentar su funcionamiento. Una de las métricas utilizadas para evaluar dicha hipótesis es la del índice de Jaccard, su función es comparar un mapa binario con un mapa probabilístico obtenido por el modelo generado y determinar la similitud entre ambos. 

Para la implementación se propuso el uso de la arquitectura \texttt{FPN}\footnote{Feature Pyramid Network}, la cuál es compatible con el codificador \texttt{ResNet}, ambos consisten en secuencias de convolución y deconvolución que realizarán las tareas de reducir las dimensiones de los datos en un un solo dato multi-dimensional y de recrear el mapa probabilístico correspondiente a las computaciones del modelo. Para validar la selección de la arquitectura y codificador se establecieron una serie de experimentos dependientes unos de otros en donde se determina la mejor configuración de parámetros para la creación y el entrenamiento del modelo así como la validación de las máscaras obtenidas mediante sobreposición de pixeles de ambas máscaras, mediante el coeficiente de dados y el criterio de Jaccard.


\bigskip\noindent\begin{tabular}{lc}
\vspace*{-2mm}\hspace*{-2mm}Firma de la asesora: & \\
\cline{2-2} & \hspace*{1em}\asesor\hspace*{1em}
\end{tabular}}

