%Resumen

\chapter{Resumen}
\markboth{Resumen}{}

{\renewcommand{\baselinestretch}{1.1}\selectfont
{\setlength{\leftskip}{10mm}
\setlength{\parindent}{-10mm}

\autor.

Candidato para obtener el grado de \grado\orientacion.

\uanl.

\fime.

Título del estudio: \textsc{\titulo}.

\noindent Número de páginas: \pageref*{lastpage}.}

%%% Comienza a llenar aquí
\paragraph{Objetivos y método de estudio:}
Desarrollar una herramienta de asistencia para la detección de cáncer de piel utilizando las técnicas mas actuales de visión computacional e inteligencia artificial, se pretende desarrollar mediante \emph{software} y tecnicas de \emph{\gls{seg}} una aplicación que permita introducir una imagen y como resultado obtengamos un mapa de características segmentado en una o más categorías. 

\paragraph{Contribuciones y conclusiones:}
\coltext{}
\newline

\bigskip\noindent\begin{tabular}{lc}
\vspace*{-2mm}\hspace*{-2mm}Firma del asesor: & \\
\cline{2-2} & \hspace*{1em}\asesor\hspace*{1em}
\end{tabular}}

