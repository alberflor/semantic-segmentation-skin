\newglossaryentry{rn}{%
	name={red neuronal},
	description={Modelo matemático optimizado mediante funciones de cálculo para dar una salida deseada en base a una entrada.}
}

\newglossaryentry{nb}{%
	name={neurona biológica},
	description={Célula fundamental para el funcionamiento cognitivo humano.}
}
\newglossaryentry{na}{%
	name={neurona artificial},
	description={Modelo matemático que simula la comunicación de las neuronas biológicas.}
}
\newglossaryentry{seg}{%
	name={segmentación semántica},
	description={Algoritmo de la inteligencia artificial que asocia una categoría o etiqueta a cada píxel de una imagen.}
}
\newglossaryentry{epidermis}{%
	name={epidermis},
	description={Se trata de la capa mas superficial de la piel, es también la mas delgada y está conformada en su mayoría por células llamadas queratinocitos o células escamosas, células basales y melanocitos.}
}
\newglossaryentry{dermis}{%
	name={dermis},
	description={Se trata de la capa intermedia de la piel, es más gruesa que la epidermis y está conformada de folículos pilosos, glándulas sudoríparas, vasos sanguíneos y nervios los cuales están sostenidos por colágeno.}
}
\newglossaryentry{hipodermis}{%
	name={hipodermis},
	description={Se trata de la capa mas profunda de la piel, la principal función de esta capa es regular la temperatura y amortiguar los golpes externos para proteger a los órganos.}
}

