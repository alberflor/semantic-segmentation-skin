
%%%%%%%%%%%%%%%%%%%%%
% Documento maestro %
%%%%%%%%%%%%%%%%%%%%%
\documentclass{fime}

%%%%%%%%%%%%%%%%%%%%%%%%%%%%%%%%%%%%%%%%%%%
% Cargando paquetes y definiendo opciones %
%%%%%%%%%%%%%%%%%%%%%%%%%%%%%%%%%%%%%%%%%%%
% Aquí puedes cargar los paquetes que vas a usar. La clase
% fime ya incluye babel, inputenc, graphicx y los de la AMS.
% Cargar un paquete está a tu libertad (y responsabilidad).
\usepackage{hyperref}
	\hypersetup{breaklinks=true,colorlinks=true,linkcolor=black,citecolor=black,urlcolor=black}
\usepackage[toc, section=chapter, nopostdot]{glossaries}
	\makenoidxglossaries	
	\loadglsentries{Glosario}
	\setglossarystyle{altlist}
\usepackage{natbib}


%%%%%%%%%%%%%%%%%%%%%
% Definiendo campos %
%%%%%%%%%%%%%%%%%%%%%
\def\titulo{Detección de cancer de piel mediante segmentación semántica}
\def\autor{Mario Alberto Flores Hernández}
\def\matricula{1719126}
\def\grado{Licenciatura en Ingeniería En Mecatrónica}
% En caso de que el grado tenga orientación o especialidad llenar el siguiente
% campo dejando un ESPACIO INICIAL, en caso contrario, dejar vacío
\def\orientacion{}
% Coloca el mes con mayúscula inicial
\def\fecha{Febrero 2021}

\def\asesor{Dra. Satu Elisa Schaeffer}
\def\revisorA{Dr. Romeo Sánchez Nigenda}
\def\revisorB{Dra. Sara Elena Garza Villarreal}
% En el caso de que tu tesis sea de doctorado activa la variable cambiándola a \doctoradotrue
% y define tus otros dos revisores
\newif\ifdoctorado\doctoradofalse
% El visto bueno siempre va
\def\vobo{Dr. Fernando Banda Muñoz}



%%%%%%%%%%%%%%%%%%%%%%%
% Inicia el documento %
%%%%%%%%%%%%%%%%%%%%%%%
\begin{document}


\frontmatter
\pagestyle{main}

\include{Portadas}

\tableofcontents


\listoffigures
\addcontentsline{toc}{chapter}{Indice de figuras}

\listoftables
\addcontentsline{toc}{chapter}{Indice de tablas}



%Agradecimientos

\chapter{Agradecimientos}
\markboth{Agradecimientos}{}
\coltext{
Aquí puedes poner tus agradecimientos. (No olvides agradecer a tu comité de tesis, a tus profesores, a la facultad y a CONACyT en caso de que hayas sido beneficiado con una beca).
}

%Resumen

\chapter{Resumen}
\markboth{Resumen}{}

{\renewcommand{\baselinestretch}{1.1}\selectfont
{\setlength{\leftskip}{10mm}
\setlength{\parindent}{-10mm}

\autor.

Candidato para obtener el grado de \grado\orientacion.

\uanl.

\fime.

Título del estudio: \textsc{\titulo}.

\noindent Número de páginas: \pageref*{lastpage}.}

%%% Comienza a llenar aquí
\paragraph{Objetivos y método de estudio:}
Desarrollar una herramienta de asistencia para la detección de cáncer de piel utilizando las técnicas mas actuales de visión computacional e inteligencia artificial, se pretende desarrollar mediante \emph{software} y tecnicas de \emph{\gls{seg}} una aplicación que permita introducir una imagen y como resultado obtengamos un mapa de características segmentado en una o más categorías. 

\paragraph{Contribuciones y conclusiones:}
La principal contribución de este trabajo de tesis está dirigido al área de \emph{hospital 4.0}, ya que el método planteado para la detección del cáncer en la piel involucra el úso de las tecnologías desarroladas mediante inteligencia artifica. Esto beneficia especialmente al proceso el paciente puede obtener su diagnostico.
\coltext{Agregar mas información.}


\bigskip\noindent\begin{tabular}{lc}
\vspace*{-2mm}\hspace*{-2mm}Firma de la asesora: & \\
\cline{2-2} & \hspace*{1em}\asesor\hspace*{1em}
\end{tabular}}




\mainmatter
\pagestyle{fime}


%%% Haz un documento para cada capítulo


\chapter{Introducción}

En los últimos años se han logrado muchos avances en cuanto al desarrollo de software inteligentes, una de las tecnologías emergentes y que están tomando gran importancia son la \emph{\gls{rn}}. Algunos de los sectores que han mostrado un incremento en el uso de ésta tecnología son: el sector automotriz (piloto automático), el sector de manufactura (optimización de procesos), el sector de entretenimiento (recomendaciones personalizadas), el sector médico (diagnóstico de imágenes). 

Este experimento tiene como objetivo la clasificación de tejidos sanos y tejidos con posible cáncer de piel (basalioma, carcinoma, melanoma) en imágenes, mediante el uso de la red neuronal de \emph{\gls{seg}} basada en el modelo propuesto en \cite{wu2019fastfcn}, con la finalidad de asistir al médico especializado en el diagnóstico de cáncer de piel a brindar atención a los pacientes con mayor probabilidad de padecer la enfermedad.

\newpage

\section{Hipótesis}
Es posible clasificar los píxeles en distintas categorías dentro de una imagen gracias a las tecnologías actuales de inteligencia artificial y las técnicas de segmentación. Mediante la técnica de \emph{\gls{seg}} es posible crear un reconocedor visual que no solo detecte la presencia y ubicación del elemento a reconocer, sino que, también obtenga otros datos descriptivos del elemento como el tamaño, forma y región que abarca dentro de la imagen. 

\section{Objetivos}
\subsection{Objetivo General}
El \emph{objetivo general} es implementar una herramienta de asistencia  para la detección de cáncer en la piel,  anualmente se registran aproximadamente 1000 casos de cáncer de piel solo en México, la detección temprana de esta enfermedad es crucial para mantener el riesgo de mortalidad al mínimo. Por lo tanto sería muy conveniente tener una aplicación que de forma automatizada pueda analizar una gran cantidad de imágenes y localizar dentro de estas la presencia del cáncer, y así optimizar el proceso de atención a los pacientes.
\subsection{Objetivo Específico}
Desarrollar el código requerido para extraer los datos de las imágenes sobre cáncer de piel, desarrollar y entrenar con dicha base de datos el modelo de la red neuronal y comparar los resultados de distintos modelos existentes de segmentación semántica.

\chapter{Antecedentes}
\chapter{Estado del Arte}
\chapter{Base de datos}
\chapter{Implementación de la solución}
\chapter{Resultados}



\appendix
%%% Haz un documento para cada apéndice, si es que tienes

\backmatter
\pagestyle{main}

% Aquí va la bibliografía, puedes usar el entorno de LaTeX (thebibliography)
% o la herramienta BibTeX. En caso de que optes por BibTeX, puedes usar
% alguno de los archivos de estilo (mighelbib.bst o mighelnat.bst) incluidos
% en el paquete, cuyos diseños armonizan con el diseño de tesis provisto por
% fime.cls. Para muestra, basta un botón:


\printnoidxglossary[sort=word]
\bibliographystyle{mighelnat}
\bibliography{MiBiblio}
\addcontentsline{toc}{chapter}{Bibliografía}

\label{lastpage}
%Autobiografia

\chapter*{Resumen autobiográfico}
\thispagestyle{empty}

\begin{center}
\autor

Candidato para obtener el grado de\\
\grado\\
\orientacion\bigskip

\uanl\\
\fime\bigskip

Tesis:\\
\textsc{\large\titulo}
\end{center}\bigskip

%Aquí va tu historia



\end{document}

