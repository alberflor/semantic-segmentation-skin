\chapter{Introducción}

En los últimos años se han logrado muchos avances en cuanto al desarrollo de softwares inteligentes, una de las tecnologías emergentes y que están tomando gran importancia son las redes neuronales\footnote {Red Neuronal: Modelo matemático que simula el funcionamiento del cerebro humano. }. Algunos de los sectores que han mostrado un incremento en el úso de ésta tecnología son: el sector automotriz (piloto automático), el sector de manufactura (optimización de procesos), el sector de entretenimiento (recomendaciones personalizadas), el sector médico (diagnóstico de imágenes). 

Este experimento tiene como objetivo la clasificación de tejidos sanos y tejidos con posible cancer de piel (basalioma, carcinoma, melanoma) en imágenes, mediante el uso de la red neuronal de segmentación semántica\footnote {Segmentación Semántica: Asociación de cada pixel dentro de una imagen a una categoría específica.} basada en el modelo FastFCN de Huikei Wu\footnote {'Fast Fully-Convolutional Network': Red neuronal convolucional desarrollada por Huikei Wu en Deepwise AI Lab (2019)}, con la finalidad de asistír al médico especializado en el diagnóstico de cancer de piel a brindar atención a los pacientes con mayor probabilidad de padecer la enfermedad.


