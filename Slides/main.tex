\documentclass{beamer}

\usepackage[utf8]{inputenc}
\usepackage{blindtext}

\usetheme{Madrid}

%Information to be included in the title page:
\title{Detección de melanoma mediante segmentación semántica}
\author{Mario Alberto Flores Hernández}
\institute{FIME - UANL}
\date{2020}


\begin{document}

\frame{\titlepage}

\begin{frame}
    \frametitle{Índice}
        \tableofcontents
\end{frame}
\section{Introducción}
\begin{frame}
    \frametitle{Introducción}
    La inteligencia artificial nos permite crear modelos que repliquen transformaciones definidas por los datos de entrada.
\end{frame}

\section{Motivación}
\begin{frame}
    \frametitle{Motivacion}
    El melanoma de piel es un padecimiento que puede ser tratado cuando es detectado a tiempo, mediante un modelo de reconocimiento visual esto puede ser realizado de forma automática.
\end{frame}

\section{Antecedentes}
\begin{frame}
    \frametitle{Antecedentes}
    \blindtext[1]
\end{frame}

\section{Estado del Arte}
\begin{frame}
    \frametitle{Estado del Arte}
    \blindtext[1]
\end{frame}

\section{Implementación}
\begin{frame}
    \frametitle{Implementación}
    \blindtext[1]
\end{frame}

\section{Experimentos}
\begin{frame}

    \frametitle{Experimentos}
    \blindtext[1]

\end{frame}

\begin{frame}
    \frametitle{Experimentos}
    \includegraphics[width=\textwidth]{example-image-a}
\end{frame}

\section{Conclusión}
\begin{frame}
    \frametitle{Conclusión}
    \blindtext[1]
\end{frame}



\end{document}
